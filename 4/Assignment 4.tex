%%This is a very basic article template.
%%There is just one section and two subsections.
\documentclass{article}
\usepackage[latin1]{inputenc} %coding of writteninput %latin1 allows for Umlaute
\usepackage[T1]{fontenc}%vectorized fonts (cm-super package)
\usepackage[german]{babel}%some specifics of the german language
\usepackage{amsfonts, amsmath, amsthm, amssymb, paralist}

  \usepackage{listings}
\usepackage{geometry}
  \geometry{a4paper, top=25mm, left=20mm, right=15mm, bottom=20mm, headsep=10mm, footskip=12mm}
 \usepackage{rotating} 
 %Decisiontree
 \usepackage{tikz}
\usetikzlibrary{arrows.meta}
  
\usepackage{graphicx} 

\usepackage{verbatim}%f�r txt datei

\usepackage{color} %red, green, blue, yellow, cyan, magenta, black, white
\definecolor{mygreen}{RGB}{28,172,0} % color values Red, Green, Blue
\definecolor{mylilas}{RGB}{170,55,241}
\usepackage{pdfpages}
\usepackage[section]{placeins}
\title{Assignment 4}
\begin{document}
 \setlength{\parindent}{0em} 
\maketitle
\section*{Exercise 1}
Let $\mathcal{X}$ be the boolean hypercube $\{0,1\}^n$ and $x\in\{0,1\}^n$. 

Let $\mathcal{H}_{n-parity}=\{h_I|I\subset [n]\} $ with $h_i(x) =
(\sum_{i_\in I} x_i) \mod 2$.

\textbf{The VC-dimension is } $\mathbf{n}$.

1. VCdim($\mathcal{H}$)$\geq n$.

Therefore let $e_i$ be the unit vectors with exactely a '1' at the i-th entry.
There are obviously n such vectors. We can now construct any bit assignment of
this vectors such that we can define 
\begin{align*}
S = \{ i| b_i = 1\}
\end{align*}
Because of that 
\begin{align*}
h_I(e_i) = \begin{cases}
1 \text{, if }i\in I\\
0 \text{ otherwise}
\end{cases}
\end{align*}

2. VCdim($\mathcal{H}$)$\leq n$ 

$2^n$ parity functions. $\Rightarrow$ VCdim($\mathcal{H}$)$\leq
\log|\mathcal{H}| = n$

\section*{Exercise 2} % (fold)
\label{sec:exercise_2}
	\subsection*{a)} % (fold)
	\label{sub:a_}
	
	% subsection a_ (end)
	\subsection*{b)} % (fold)
	\label{sub:b}
		Proof idea:
		
		\begin{itemize}
			\item Set $X=\{a,b,c,d,e,f,g,h\}$
			\item $\mathcal{H}_1$ shatters $\{a,b,c\}$
			\item $\mathcal{H}_2$ shatters $\{d,e,f\}$
			\item Construct $\mathcal{H}_1,\mathcal{H}_2$ s.t. $\mathcal{H}_1\cup \mathcal{H}_2$ shatters $X$
			\\ but in a way that the VCdims of each one alone is still 3 
			\\ e.g.: $\mathcal{H}_1=\{h_a,h_b,h_c,h_{ab},h_{bc},h_{ac},h_{abc},h_g,h_h\}$
			and $\mathcal{H}_2=\{h_d,h_e,h_f,h_{de},...,h_{gh},...\}$ (i.e. everything that's missing to shatter $X$) 
			\item Then $\mathcal{H}_2$ also shatters $\{d,e,f,g\}$ since $h_{gh} \equiv h_g$ when applied to this subset
			\item It should be possible to expand this argument to arbitrary sets
		\end{itemize}	

	% subsection b (end)
% section exercise_2 (end)

\section*{Exercise 3}
$\mathcal{H}^d_{con}$ is the class of Boolean conjunctions over the variables
$x_1,\ldots,x_d$.
\subsection*{a)}
$\mathbf{|\mathcal{H}_{con}^d|}\leq 3^d + 1$

%http://citeseerx.ist.psu.edu/viewdoc/download?doi=10.1.1.54.8523&rep=rep1&type=pdf

% Foundations of Machine learning p. 18 ff

The variable $x_i$ is either positive, negative or not included. So the
hypothesis space $\mathcal{H}$ is at most of size $3^d + 1$ (probably $3^d$). So
like in Exercise 1 we can conclude:

VCdim($\mathcal{H}_{con}^d$)$\leq \log|\mathcal{H}_{con}^d| = d*log(3)$

\subsection*{b)}
% Proof idea:
% 
% We can represent any set $X=\{e_{i_1},...,e_{i_j}\}\subseteq \{e_i|i\leq d\}$ by omitting $x_{i_1},...,x_{i_j}$ from the function and get $\overline{x}_1\land \ldots \land \overline{x}_{i_1-1} \land \overline{x}_{i_1+1} \land \ldots \land \overline{x}_d$

If you assign every unit vector $e_i$ a variable $x_i$ and $\bar{x_i}$, than if
you want to have $e_i$ in the subset you choose $x_i$ in the conjunction and
otherwise $\bar{x_i}$. This only makes true boolean conjunctions possible. 

\subsection*{c)}
%http://citeseerx.ist.psu.edu/viewdoc/download?doi=10.1.1.54.8523&rep=rep1&type=pdf
Let $S\subseteq \{0,1\} ^ n$ be a shattered set of size $d+1$. 

Fix an enumeration $u_1,\ldots,u_\{n+1\}$ of the elements of $S$. Let $S_i$ be
$S\\ u_i$. The definition of shattered gives, that for each $S_i$ exists a
conjunction $c_i$, such that $S_i = S\bigcap c_i$.

%offical bla bla 
We can conlcude, that $c_i$ is false on $u_j$, if $i=j$.

So every $c_i$ must contain a literal $l_i$, that makes it false on $c_i$ and
true on $c_j$. Because of the size of $S$ at least one literal occurs twice.

Wlog this literals are $l_k,l_m$. 

There exists two cases:
\begin{enumerate}
  \item $l_k=l_m$\\
  \item $l_k=\neg l_m$
\end{enumerate}
In the first case $l_k$ would be false in $u_k$ and $u_m$, but then $c_k$ is
false on both $u_k,u_m$.

In the second case we have that for $u_n\neq u_k,u_m$, $l_k \text{ or }\neg l_k$
has to be false, but than either $c_k \text{ or } c_l$ is false on $u_n$.

\subsection*{d)}
%https://ac.els-cdn.com/0020019096000841/1-s2.0-0020019096000841-main.pdf?_tid=7e70d894-4d2c-4da7-be90-e60f3fd8fa26&acdnat=1525862694_d205ac3414c033b4e77e7b255ee3c056

%http://citeseerx.ist.psu.edu/viewdoc/download?doi=10.1.1.54.8523&rep=rep1&type=pdf



\section*{Exercise 4}
Observing a set of variables in $\{0,1\}^n$ we see, that there are $n+1$ different variables for a symmetric function which are different in the amount of 1's: \\ ${(0,0,..., 0), (1,0,..., 0), (1,1,..., 0), ... (1,1,1, ... 1)}$ \\
Let $ones(x)$ be a function which returns the number of ones in x and let $\phi_i(x)$ be a symmetric function (such that $i\in\{0,..., n\}$) which labels input $x_i$ with $1$ and is defined as follows:
$\phi_i(x) = (ones(x) = i)$.
Then for any labeling we can find a symmetric function $\psi$ which is a composition of different  $\phi_i(x)$'s by disjunction. \\
So $VCdim(\mathcal{H}) = n+1$ as for observed symmetric functions only $n+1$ different inputs are possible.

\end{document}
