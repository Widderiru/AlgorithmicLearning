%%This is a very basic article template.
%%There is just one section and two subsections.
\documentclass{article}
\usepackage[latin1]{inputenc} %coding of writteninput %latin1 allows for Umlaute
\usepackage[T1]{fontenc}%vectorized fonts (cm-super package)
\usepackage[german]{babel}%some specifics of the german language
\usepackage{amsfonts, amsmath, amsthm, amssymb, mathabx, paralist}
 \setlength{\parindent}{0em} 
  \usepackage{listings}
\usepackage{geometry}
  \geometry{a4paper, top=25mm, left=20mm, right=15mm, bottom=20mm, headsep=10mm, footskip=12mm}
 \usepackage{rotating} 
 %Decisiontree
 \usepackage{tikz,forest}
\usetikzlibrary{arrows.meta}
  
\usepackage{graphicx} 

\usepackage{verbatim}%f�r txt datei

\usepackage{color} %red, green, blue, yellow, cyan, magenta, black, white
\definecolor{mygreen}{RGB}{28,172,0} % color values Red, Green, Blue
\definecolor{mylilas}{RGB}{170,55,241}
\usepackage{pdfpages}
\usepackage[section]{placeins}

\title{Assignment 1}

\begin{document}

\section*{Exerise 1}
\subsection*{a)}
The VC-dimension is 3, because you can shatter all sets of size 3, but it is not
possible to shatter the set $\{+-+-\}$. Like to see in the following picture:

\includegraphics[width=.6\linewidth]{1a.jpg}

\subsection*{b)}

The VC-dimension is $\infty$, 

You construct the set as following:
\begin{enumerate}
  \item Use only primenumbers as basis for construction $B=\{p_1,\ldots,p_k\}$
  \item As set $X$, you want to shatter, use the product of the numbers in
  $B/p_i$ for a $p_i$
  \item The $p_i$, that you doesn't used, is the greatest common divisor of the
  other numbers
  \item If you want to shatter such a set, than the primenumbers you do not have
  in the products are the $k$ you need to use.
  \item This construction is possible for $\infty$ many combinations
\end{enumerate}
\begin{tabular}{|c|c|c|c|c|}
	  &$p_1$ &$p_2$ &$p_3$ &$p_4$\\
$x_1$ &		 &1		&1	   &1\\
$x_2$ &1	 &		&1	   &1\\
$x_3$ &1	 &1		&	   &1\\
$x_4$ &1	 &1		&1	   & \\
\end{tabular}
\section*{Exercise 2}
The first observation is, the points have to be in a circle. Wouldn't they be in
a circle, there would always be a point in the middle or a little bit outside,
that destroys the shattablilty. The points outside and the point in the middle would have to be of
the same ``classification''. Also is the class of the point a little bit outside
directly defined, because of the points closer to the rest.

It is possible to sort 7 points such that they are shattable. Subsets of size 1
and 2 are trivial. All other combinations you can see on the picture. The circle
is symmetric, so you can turn the picture in all directions and it will still
works out.

\includegraphics[width=.6\linewidth]{2.jpg}

If you try for 8 points, this construction will fail already with a subset of 3.
We know because of the beginning, that they have to be in a ring organized. If you
have that than the following is a counterexample of a subset. You can try to
draw the points a little bit different, but because of the symmetry you will
always have the same problem for a subset. 

\includegraphics[width=.6\linewidth]{22.jpg}

\end{document}
